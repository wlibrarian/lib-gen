\chapter{Contra `The case for colonialism'}

\textbf{Joshua Loo}

\begin{multicols}{2}

The reëvaluation on both sides of the is-ought distinction of colonialism
proposed in `The Case for Colonialism'\footcite{gilley} relies on a number of flawed assumptions and models which do not, as is purported to be the case, justify the conclusions that Gilley draws, viz. that it was `objectively beneficial and subjectively legitimate in most of the places it was found, using realistic measures of those concepts.' It suffers from a deficit of ethical reasoning, which is particularly disappointing given the strong ethical claims it makes as to the nature of colonialism, and conflates the falsity of certain views with potential intellectual failings on the part of their proponents.

Gilley objects to anti-colonial claims in three areas---`objective harm', `subjective \ldots illegitima[cy]' the violation of `the sensibilities of contemporary society'.

The main mechanism by which the desired `objective costs/benefits approach' is to be effected is the `counterfactual'; `what would likely have happened in a given place absent colonial rule,' he asks. There are seven major problems with the general and particular use of the counterfactual as moral evaluation.

First, it is curious that he should have limited his work to the period between the early nineteenth and mid-twentieth centuries; had colonialism truly been of negligible impact before the nineteenth century, it would have been relatively trivial to demonstrate this; were it more important than Gilley acknowledges, it would represent a significant oversight.

The problem in this case is that Gilley has completely omitted a significant part of colonial history. The historical and moral question of whether colonialism was a good thing presumably seeks to evaluate what happened; it should not, therefore, evaluate an idealised colonialism---one in which some of the most significant maleffects thereof have been ignored in what one hopes is a coïncidence. An evaluation of colonialism, therefore, should certainly not fail to consider the Atlantic slave trade, in which `a cumulative total of over 10 million Africans reached the New World as slaves from 1500 to 1900; closer to 12 million were dispatched in ships from Africa, and over 1.5 million lost their lives in the middle passage. In the same period, some 6 millions slaves were sent from sub-Saharan Africa to the Orient, and some 8 million people were enslaved and retained within the African continent. An estimated total of 4 million people lost their lives'\footcite{slavery}. Even in a world of billions we consider genocides of tens of thousands to be abhorrent. When considering the scale of this trade, we should remember that the vast majority of it occurred before the middle of the nineteenth century, when the British curtailed it; the population of the world stood at one billion in 1804 and two billion in 1923\footcite{pop}; for each slave there were many more born into slavery, and so on.

Second, there was no counterfactual. Gilley at no point details how he thinks countries would have developed without colonialism. The assumption that `pre-colonial histories ... \ldots by definition \ldots [possessed] comparatively weak institutions' is simply false. If Gilley is truly unaware of the development of governmental institutions in, for example, China, such a lack of knowledge should disqualify him from being an academic; at best, this is an oversight, and at worst, the extensive history of political development of non-Western societies was ignored to prop up a provocative conclusion, with no attempt to find what really happened. It is difficult to escape the latter conclusion when one has read the call to rigour and `epistemic virtue' in the coming pages.

Third, to the extent that there is a counterfactual, Gilley's counterfactual does not control for a number of factors which would likely not have existed without colonialism. Although he considers this possibility---`[c]ountries that did not have a significant colonial history---China ... and Guatemala, for instance---provide a measure of comparison to help identify what if anything were the distinctive effects of colonialism', one should consider why many of these countries were not invaded: many of them lacked the riches necessary to be worth invading, others may have been too strong to be worth the resources required for their invasion, others still were skilful in their manipulation of Western powers to suit their interests and so on. Equally, colonialism can affect countries which were not directly colonised; most importantly they may have relied on formerly independent nations for trade, which could have been affected by incoming colonisers; they may have had to divert resources to their own defence, at the expense of development; they may have been alienated from modernity, and consequently from the institutions which the West developed which may have uplifted their people, by their having witnessed what occurred in other countries, and so on.

Fourth, again, to the extent that there is an attempt at a counterfactual, Gilley's attacks on anti-colonial criticism of colonial governance on the grounds that the gravity of the problems faced by colonial powers was insufficiently considered ignores the possibility that these problems would not necessarily have existed without colonialism. Gilley is defending here the whole of colonialism; it is of course unlikely that such a phenomenon of this size should not have benefited a single person other than those who were its primary instigators, but that is insufficient to answer the overall question Gilley has asked himself.

Fifth, any attempt to create a counterfactual is intrinsically flawed. Even if one is not a determinist, one must still recognise that for colonialism not to have happened, many things in the world before must have been different. The sort of change necessary to effect such a change must be sufficiently large to have either reduced the European advantage (perhaps they could be stripped of their guns?) or decreased non-European disadvantages (perhaps they could have adopted guns more?). These would have changed the course of history in a way which is almost impossible to predict. It is not, therefore, possible to measure `training for self-government, material well-being, labor allocation choices, individual upward mobility, cross-cultural communication and human dignity' in the situation which `would likely have obtained': historians may argue with some certainty about what would have happened were one thing to have been changed, but, even one who is 90 per cent confident in their abilities would, over the period of a century, find themselves far closer to no confidence at all than what is required of this exercise. 

Sixth, even were the vast intellectual and epistemic obstacles to the construction of a counterfactual to have been overcome, it is not clear that Gilley has established that the counterfactual necessarily ought to be used as a comparison. When a crime is committed, it is often quite possible that the crime could be of net positive utility. We nevertheless do not refrain from punishment and condemnation, for a number of reasons: it is not clear how we ought to measure what positive utility is, it is not certain whether the crime was of positive utility, and, most importantly, we consider some actions to be forbidden, even if they happen to be of positive utility. Perhaps, for example, one could construct a counterfactual in which the creation of the State of Israel in the present reality, as caused by World War II, ensured the survival of the Jewish people (not that this is true). Yet we shall always consider the Nazis to have been evil, abhorrent, and, hopefully, aberrant entities; we should not change our opinion for any counterfactual. Similarly, the crimes of colonialism---the massacres, the stealing, the routine violation of human dignity in even the most mundane of colonised-coloniser interactions, and so on---ought not necessarily to be forgiven even if the counterfactual is as Gilley says.

Seventh, Gilley writes that the `objective costs/benefits approach identifies a certain need of human flourishing - development, security, governance, rights etc.'. Gilley here is making a moral claim; `costs' and `benefits' are normative terms, and so must be treated with the caution that they deserve. There is a limited attempt to illustrate how one could determine what these costs and benefits actually are: Gilley continues that in `a brutally patriarchal society, for instance, access to justice for women may have been more important than the protection of indigenous land rights (which may be part of that patriarchy), as Andreski argued was the case for women in northern Nigeria under colonialism'. This individual example is indeed compelling; yet colonialism was not always progressive---the \textit{foyers socials} come to mind---and so not examples were as stark as these.

Let us now consider the second consideration---`subjective illegitimacy'.

First, we ought to question the use of the word `subjective'. It is true that illegitimacy, as a concept, is difficult to ground. Yet almost all other moral and ethical concepts are difficult to ground. There will always be a `why?' which cannot be answered. Even logic is vulnerable to questioning: why is it that \(x \equiv x\). There are no `objective' costs and benefits; these all rely on value judgements which cannot be justified after a certain point. Gilley therefore creates a false dichotomy between illegitimacy and costs and benefits. Illegitimacy may itself be a cost or a benefit; it may be contrary to what one needs for `human flourishing'. That humans happen to consider some things to be required beyond that required for survival---indeed, that humans consider survival to be a desideratum---is no more absurd when it manifests in a desire for good governance than it is when it manifests in a desire for legitimate or traditional governance.

Second, Gilley claims that `[m]illions of people moved closer to areas of more intensive colonial rule, sent their children to colonial schools and hospitals, went beyond the call of duty in positions in colonial governments, reported crimes to colonial police, migrated from non-colonised to colonised areas, fought for colonial armies and participated in colonial political processes---all relatively voluntary acts.' There are a number of potential explanations for all these phenomena which are all far more plausible than the claim that colonialism was legitimate.

First, colonisers were likely to have selected the areas most friendly to human habitation---those areas which were most fertile, with the best geographic and infrastructural connexions to the rest of the world, with the most existing infrastructure and so on. These alone explain why many would have chosen to move to more intensively colonised areas.

Second, non-colonised areas, traumatised by the experience of colonisation, could well have moved away from the beneficial aspects of (initially) European modernity: sewerage, mass education and so on. They would not have done in a non-colonial counterfactual to the same degree that they did. A colonised area with a little democracy and a relatively consistent dictatorship in the guise of a system dedicated to the `rule of law' may seem superior to another area governed by a leadership who have associated modernity with the hated European.

Third, the nature of colonialism is that it replaces alternatives, explaining much interaction with colonial systems. That an Indian chose to report a crime to a colonial policeman may not have been because the Indian thought that the British had a superior policing system to what had existed before or what would have existed; it simply reflects the inability of the Indian to summon people from the past to arrest a wrongdoer. The choice facing the colonised was often between nothing and a colonial option. Similarly, we should not be surprised that many assisted what was the only government they had; colonialism does not change the fact that in many cases coöperation with the government is in one's interest, whether by joining the army or working in the civil service.

Fourth, in a world of billions, it is unsurprising that millions should have chosen to go one way or another. We would not think Iraq a prosperous or safe place at present simply because some people choose to move there; millions are comparatively few given the time scale.

Fifth, we equally underestimate the power of the coloniser to convince the colonised of the need for colonisation. Gilley would have done well to ask himself who had a monopoly on education, force, and modern technology at the time. To a child in a rural village in the early twentieth century, the motor-car must have seemed quite something. It is unsurprising, therefore, that one who has not directly been victimised, but whose experience has simply been distant observation of God-like prowess in the manipulation of the environment and of the forces of nature for human needs, should not always resist the coloniser. 

Finally, let us consider Gilley's third criticism of anti-colonial thought. Gilley here confuses problems in the advocacy of a certain view with problems with that view; if the logic that some who oppose colonialism have said or acted in problematic ways implies the falsity of this view were true, Gilley himself would be wrong, for there have been at many times pro-colonial leaders who have committed deeply terrible acts.

The second section of the article contains more truth. It is true, for example, that anti-colonial and nationalist leaders have often committed grave sins, and have reversed whatever progress was achieved under colonial rule. There is, however, an implicit conflation between the view that colonialism as a whole was a negative phenomenon and the view that the reaction to colonialism was often unhelpful.

Gilley may, for example, be correct that Guinea-Bissau's `anti-colonial ``hero'' Amilcar Cabral' greatly harmed his people. Yet we should ask how he came to be: it was not anti-colonialism which created Cabral, but colonialism; it was not the reaction of a people long suppressed which was the ultimate cause, but rather the original actions which caused their reaction. 

In the general case, therefore, it may in fact have been true that colonialism, having destroyed the capacity of states to independently act, may have been needed to solve the problems of its own making. That is no justification for colonialism, though it may be justification for neo-colonialism.

It is strange that Gilley should therefore, having noted the costs of anti-colonialism, advocated even more of the cause of all the problematic behaviours he describes; it is strange that he should wish for more colonialism when it has prompted relatively liberal post-colonial countries such as India and Brazil to ignore abuses in their fellow post-colonies; it is strange, further, that when colonialism almost always requires the destruction of its opposition's (that is, local) capacity to act, he laments failures of governance and the loss of state capacity in postcolonial nations.

Gilley seeks to claim many effective structures for the colonial cause; ` the colonial governance agenda explicitly affirms and borrows from a country' s colonial past, searching for ideas and notions of governmentality.' There is nothing intrinsically colonial to many of the ideas he espouses. That in the past another group may or may not have chosen to act in a certain way does not mean that one should have to associate any replication thereof therewith. The Romans built roads; we do not call the road-building agendum Roman; the Mughals built palaces, yet we do not call the maintenance of the Queen's palaces Mughal, though Mughal palaces are far superior to anything of Her Majesty's; there is intrinsic in the `good governance' agendum nothing which, having been the product of British or French genius, could not also have been created by us, Afro-Asian primitives though we are.

What Gilley says near the end of the article is provocative, but also potentially true. It may have been that colonialism has so much stunted some countries that they cannot help themselves out of the whole which the Europeans have dug. Yet Gilley's decision to call such projects `colonial' ignores that one is a rectification of the damage which the other did; one explicitly seeks to construct, and to aid construction, whereas the other sought to monopolise, and to prohibit the independent development of the capacity to construct. Perhaps near the end of colonial rule, we witness in the plethora of Legislative Councils and efforts to localise civil services an attempt at such rectification something which comes close to this division of colonialism, yet we should note that most of these efforts were caused by a desire to avoid the embarrassment which would have ensued were ex-colonies to have immediately collapsed, and a sense of guilt at a lack of previous preparation. 

Some have sent death threats to Gilley; others have sent death threats to the Editor. Even if they should deserve such death threats, which they do not, for, were one to deserve death threats for poor scholarship, we should all deserve to live in fear, this is a deeply counterproductive move. Already, reporting on the incident has focused on the death threats, instead of the many criticisms of the paper, of which this article is but one. If the general public should become convinced that academia has lost the ability to critique and criticise, it will almost certainly lose whatever respect it has for the still-functional intellectual apparatus thereof, an apparatus which is still useful in searching for truth, even if its progress should have been stunted in recent years in certain fields by a lack of funding and a closing of the Overton window. 

However, Gilley and the \textit{Third World Quarterly} themselves are partly to blame. They do not deserve the death threats, and should not be blamed for that. They should, however, be blamed for their poor scholarship. It is difficult to believe that Gilley should have been so ignorant of basic moral philosophy or history as to publish what he did. It is shocking to discover the poor quality of the peer review process---quite how the paper was published is unclear, and a story which on its own ought to be heard. Finally, and most sadly, it is, again, difficult to avoid the conclusion that the article was published in bad faith without concluding instead that Gilley is guilty of many academic and intellectual failings. There is, sadly, a plausible reason for his having published such a poor article: predictably, press coverage has largely focused on the death threats, and so, the same narrative which has perpetuated the idea of a crackdown on freedom of speech at universities, convinced many that there are no arguments but only threats for leftist ideology, and has conflated attacks on ideas with attacks on the permissibility of their expression, will continue its march. One day, Gilley et al. will be the victims of such tactics; it may only be then that those who choose to take this path realise that it also will hurt them too, but by then, it will be too late to halt. 

\end{multicols}