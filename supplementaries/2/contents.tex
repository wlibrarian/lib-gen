\chapter{Miscellany}

\begin{multicols}{2}

\setlength{\parskip}{0em}
\tableofcontents
\setlength{\parskip}{1.2em}

\vspace{1.2em}
\hrule{\linewidth}

\vspace{0.55em}

Letters do not necessarily reflect the views of \textit{The Librarian}, the editor thereof, the Library Committee, the school, the library, the librarians or anyone else. Our present, tentative, policy is to publish all letters which are not defamatory and do not incite violence. We do not accept responsibility for the grammar, syntax and orthography of letter-writers; consequently, letters are published as they were sent.

\textsc{Sir},---
While I was very much heartened to see the diæresis resurging in its usage, I remain disappointed by your use of the unclear annoyance that is the regrettably standard notation for the dental fricative, viz. "th", when one could use the neater symbols Eth ("ð") and Thorn ("þ"). You ought also to inspect possible alternatives to "sh", such as "š”, which are much more phonetic.

\vspace{-1.2em}
\hspace{0.1\linewidth}I am, \&c.,
\vspace{-1.2em}

\hspace{\fill}\textsc{Nanni}

\textsc{Sir},---
If one were to list all the ways in which the ‘Library News’ section falls short of the high standards that have come to be expected of the Librarian as a publication, that list could perhaps fill hundreds of your (magnificently typeset) pages. In short, the ‘Library News’ section is the most appallingly written nonsense I have had the misfortune to encounter. For the sake of the continued sanity (such as it is) of your readers, I humbly ask you remove the section as it currently stands and replace it either with accessibly recreational mathematics, or something written by someone literate.

\vspace{-1.2em}
\hspace{0.1\linewidth}I beg to remain yours, \&c.,
\vspace{-1.2em}

\textgreek{\hspace{\fill}\textsc{ουδεις}}

\columnbreak

This issue's cover is a rather æsthetically pleasing rendering of a tiling of the Poincaré disk with triangles. Isky Mathew's latest installation in the Adventures in Recreational Mathematics series explores the world of non-Euclidian geometry---geometry which leaves the `{``}flat'' geometry of everyday intuition' and occurs on hyperbolic or elliptic planes\footcite{wolframalpha}. The Chair regretfully informs me that there is nothing to publish in the library news section.

Benedict Randall Shaw reintroduces readers to areal coördinates on page 11. `Ideally, one ought not to use areals, as the vast majority of geometric problems have neater, more satisfying and easier solutions using standard Eucildian methods,' he writes. `However, if one is unable to use those effectively for some reason, and has tried and failed to rectify this, then areals are a a possible alternative; one must, however, practise using them if one wishes to use them properly and efficiently.' His original introduction was `below the usual standard and worryingly cursory', hence our second visit to the area.

He further writes, in `On Thermoacoustic Refrigeration', of the promise of the same; these notes were taken in a lecture at Huxley Society delivered by James Tett. Conventional refrigerators must use `nasty chemicals' (refrigerants), and must consume much power; thermoacoustic refrigerators avoid these problems, and, additionally, have only one moving part, viz. the loudspeaker. The problem, unsurprisingly, is expense; this is because some of the parts, such as the `alternator required by the loudspeaker, are uncommon'. 
	
We reprint Benedict Mee's Locke essay, on Isocrates' and Plato's teaching of rhetoric, by his kind permission. 

``Contra `The case for colonialism''' considers Bruce Gilley's paper `The case for Colonialism'\footcite{gilley}, noting a number of flaws in Gilley's counterfactual approach, the lack of ethical reasoning to support his strong ethical claims, and the general furore surrounding the publication of the paper.

The substantive portion of the issue is concluded by Jonathan Watt's Sonata in D; we print here several pages thereof, being the second and third movements.

We accept articles, agony aunt questions, errata, letters\footnote{We have recently received our first letter, which is most exciting.}, short stories, poems, pieces of music and anything else which meets the standard in our Editorial---`gratification of one's intellectual curiosity'. One can contact the editor at \texttt{joshua.loo@westminster.org.uk} or \texttt{j@joshualoo.net}. We may accept anonymous pieces. We would also likely notice anything slipped under the door of room five in the lower corridor of College. \textit{The Librarian} is typeset in \LaTeX{}. \texttt{fbb} is used to provide kerning tables and the font.

Articles do not necessarily reflect the views of \textit{The Librarian}, the editor thereof, the library, the Library Committee, the school, or the author\footnote{It is rare for articles contradicting the view of the author to be submitted.}.
	
\end{multicols}