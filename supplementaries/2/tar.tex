\chapter{On Thermoacoustic Refrigeration}

\hspace{\fill}\textbf{Benedict Randall Shaw}'s notes on \textbf{James Tett}'s lecture.

\begin{multicols}{2}
	
These notes are part of an ongoing series of lecture reviews. To increase ease of comprehension and concision, they are not written using indirect or direct speech; rather, that which was said is simply replicated. \textit{The Librarian} does not endorse the content of any particular lecture.

Commentary, where provided, will be in \textbf{bold}. However, in these notes, \textbf{terms in bold} are to be defined immediately after; we use this convention here temporarily, and it may not be repeated.

\paragraph{Thermoacoustics} This should be fairly intelligible to anyone with a non-zero history of dealing with English words; being a combination of the prefix \say{thermo-}, meaning \say{relating to heat}, and \say{acoustics}, meaning \say{relating to sound}. The rough meaning of thermoacoustics should therefore be fairly apparent.

\paragraph{Sound} As anyone who has paid any attention through Fifth Form Physics will probably know, sound is composed of compressions and rarefactions of the medium through which the sound is travelling. However, Gay-Lussac's law tells us that, for a given mass and constant volume of an ideal gas, the pressure of said gas is directly proportional to its temperature in Kelvin. It therefore follows that, as sound is composed of oscillations in pressure, it must also be composed of oscillations in temperature. This is crucial to thermoacoustic refrigeration, as we shall go on to see.

\paragraph{Standing wave} A standing wave is a wave with a phase velocity of zero; that is to say, the peaks of the wave do not move spatially; rather, they simply oscillate in place. They are produced when two waves of equal amplitude, travelling in opposite directions, meet and overlap.

\paragraph{Refrigerator} While we commonly think of a refrigerator simply as being an appliance that makes an area cold, it does not do this by magic. Rather, what a refrigerator does is that it moves heat from one area to another, thus making the former colder and the latter warmer. Traditionally, the area made warmer is just that in which the refrigerator is.

One induces a standing wave of sound within a device called a resonator, which is essentially a long tube with some apparatus in, not by independently producing two waves of sound, but by producing one ordinary wave of sound using a loudspeaker at one end, and having the sound wave produced by it hit a hard surface at the other, which causes the wave to reflect back and overlap with itself, thus producing a standing wave. Note that being a sound wave, it therefore intrinsically involves oscillations in temperature and pressure.

The sound must be travelling through some medium; this is known as the working fluid (usually gaseous). Note that packets of the fluid expand when warm and compress when cold. As the temperature is oscillating in places due to the standing wave, there are therefore packets of air which become hot, and so expand and release heat, move forward, become cold, and so compress and absorb heat, move back again, and then repeat. This means that the packets of air absorb heat in one area and release it in another, and thus induce a temperature gradient in the surface to/from which they are releasing/absorbing heat.

To maximise this surface area, we use an item called a stack, which consists of many parallel channels. The standing wave therefore makes one end hot and one end cold; to use this, we put a microchannel heat exchanger at either end, which conducts heat to the cold end and away from the hot end. These heat exchangers have no moving parts.

We therefore have a mechanism that moves heat from one place to another, which is what we wanted.

\paragraph{Loudspeaker} We need the loudspeaker to be sufficiently sonorous to produce a wave of 180--200dB.

\paragraph{Working fluid} We want the working fluid to have a low viscosity to avoid inefficiency from friction, a high specific heat capacity to aid transfer of energy, and a thermoconductivity that is not too high, lest the temperature difference in parts of the fluid decrease as a result of heat moving from hotter areas to colder, but not too low, in order that heat may be absorbed from and released to the stack.

\paragraph{Stack} We want the stack to have a high surface area, to aid the transfer of heat between it and the fluid. We also want it to have a high specific heat capacity and a low conductivity, in order that the temperature gradient remains in place; that is to say, in order that the heat stays where we put it using our standing wave.

Thermoacoustic refrigerators have a number of advantages over those currently in household use, which have numerous problems: sliding seals are used, and so whatever manufacturers do, a small amount of refrigerant (the liquid used in the mechanism) will leak; they use a lot of power; and refrigerants themselves are nasty chemicals, often being flammable and toxic, and being harmful for the environment.

Thermoacoustic refrigerators avoid these problems to a large extent. The working fluid need not be toxic, and in fact can be a gas such as nitrogen, which forms \(78\%\) of the air and is non-toxic and inert (thus leakage does not matter); they are sufficiently efficient that they are used in space and by navies; and they only have one moving part (the loudspeaker), and so don't leak things.

At the moment, thermoacoustic fridges are expensive, as certain parts, such as the alternator required by the loudspeaker, are uncommon. However, they are starting to be used by the civilian population; for example, \(40\%\) of \textit{Ben and Jerry's} stores now use thermoacoustic refrigerators.
\end{multicols}