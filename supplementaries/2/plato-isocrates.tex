\chapter{How far do Isocrates and Plato agree about the teaching of rhetoric?}

\textbf{Benedict Mee}

\begin{multicols}{2}

By 387 BC, Plato had founded his famous Academy in Athens.\footnote{\textsuperscript{}
	Benoit, \emph{Isocrates and Plato on Rhetoric and Rhetorical
		Education}, 1991} Less well known today is another school, established
a few years earlier, probably around 392 BC, by Isocrates.\footnote{\textsuperscript{}
	Plutarch, \emph{Lives of Ten Orators}} While one of these teachers and
his school remain renowned even today, the other has faded into relative
obscurity. But it was not always so. Dionysius of Halicarnassus praised
Isocrates as `outstanding among the famous men of his day and the
teacher of the most eminent men in Athens and Greece'.\footnote{\textsuperscript{}
	Dionysius of Halicarnassus, \emph{Isocrates}} To Cicero, Isocrates was
`master of all rhetoricians'; in the eyes of Quintilian `we owe the
greatest orators to the school of Isocrates'.\footnote{\textsuperscript{}
	Cicero, \emph{De Oratore}; Quintilian, \emph{Institutio Oratoria}} By
the mid fourth century BC, Plato's Academy and Isocrates' school were
the two most prominent educational institutions in Athens.\footnote{\textsuperscript{}
	Grote, \emph{Plato and Other Companions of Socrates, 1867}} When
Plato's pupil Aristotle began criticising Isocrates in his lectures at
Plato's Academy, Isocrates' pupils Kephisodorius and Theopompus
retaliated, taking aim at Plato's discourses.\footnote{\textsuperscript{}
	Ibid} While there has been some debate on the depth of the hostility
between Plato and Isocrates, it is most likely that, as their schools
grew in parallel, a rivalry developed between these two eminent Athenian
educators, as well as their pupils. This rivalry is reflected in their
works, in which at various points each takes on the other (with varying
degrees of subtlety).\footnote{\textsuperscript{} Benoit, 1991} These
works reveal a shared concern for the place of a rhetorical education in
life, and a common approach touching on the relationship between an
education in speaking and truth, its impact on the soul, and who the
most successful students of oratory would be. Nevertheless, the two had
differences in outlook which are not to be ignored. Plato has gone down
in history as a `philosopher', and fittingly he was much more concerned
with philosophy than rhetoric, whose place in education (and indeed life
itself) he felt should be secondary to philosophy. Isocrates approached
the issue from the other direction, indeed he took oratory to be central
to education, philosophy and life as a citizen.

Outwardly, Plato and Isocrates appeared to espouse very different views
of an educational curriculum, and rhetoric's place in it. Pupils at
Plato's Academy were taught a wide curriculum which centred on
philosophy and mathematics (tradition has it that the motto `let no one
unschooled in geometry enter' was inscribed above the entrance -- the
breadth of study was literally lapidary). Isocrates, on the other hand,
looked down upon other areas of study. While he admitted they might be
useful training for the mind, he felt subjects like geometry were of
little lasting benefit.\footnote{\textsuperscript{} Isocrates,
	\emph{Antidosis} (331-333)} Students of these kinds of things gained
no practical skills. That does not mean, however, that Isocrates thought
that acquiring skill in rhetoric was the sole purpose of an education.
His curriculum was certainly broader than a simple check-list of
speaking techniques. Not only did Dionysius of Halicarnassus say of
Isocrates' pupils that they were distinguished in politics, public life,
and even as Historians; Isocrates positioned himself as a teacher of
quite a broad area.\footnote{\textsuperscript{} Dionysius of
	Halicarnassus, \emph{Isocrates}} He positioned himself as a rhetorical
teacher in opposition to prevailing methods in his manifesto
\emph{Against the Sophists}, published shortly after his school was
founded. In it, he levelled three main criticisms of the sophists'
curriculum. Firstly, Isocrates said that they simply doled out analogies
for their pupils to learn and apply, unsuited to the circumstances of
individual speeches.\footnote{\textsuperscript{} Isocrates,
	\emph{Against the Sophists }(13)} Secondly, he criticised teachers of
rhetoric who did not examine the nature of the knowledge they imparted,
with the result that their pupils parroted what they had learned by
rote, rather than thinking for themselves.\footnote{\textsuperscript{}
	Isocrates, \emph{Against the Sophists }(9-10)} The third criticism
Isocrates levelled at these sophists was that they had `no interest in
truth' in what they taught.\footnote{\textsuperscript{} Ibid}
\emph{Against the Sophists} was a manifesto for his teaching, designed
to distinguish himself from his competitors and advertise his own
methods of teaching.\footnote{\textsuperscript{} Benoit, 1991} Clearly,
his own pupils could expect a course of studies which taught them not
only to think individually and originally, but develop an understanding
of both `the nature of knowledge' and the truth behind how they were
being taught to speak. Isocrates set out then, to teach his pupils the
foundations of the subjects on which they might speak. Moreover, in
\emph{Against the Sophists}, Isocrates is also critical of the sophists'
philosophical teachings, and takes particular aim at the
\emph{Eristics}, who taught ethics.\footnote{\textsuperscript{}
	Isocrates, \emph{Against the Sophists }(1-3)} So, given all this and
the topics of his own surviving orations -- as well as the common
applications of rhetoric in the assembly and the law courts -- it is
plausible that Isocrates' curriculum could have included History and
Politics as well as perhaps even some Philosophy. We cannot know for
sure what the curriculum at Isocrates' school was, just as we are unsure
of the Academy's precise syllabus. What is clear, however, is that while
Isocrates' view of education prioritised rhetoric more than Plato's
Academy, and definitely did not teach mathematics, it was not as unlike
the Academy as it might first seem. The broader educational context was
crucial to both Plato's and Isocrates' approaches to teaching rhetoric.

Isocrates' concerns with other rhetorical instructors' disregard for
truth (mentioned above) were very much shared by Plato. If anything,
Plato was more vehement than Isocrates in his criticism of what he saw
as sophists' dishonesty. Plato repeatedly complained that in public
speaking `what is true' had been replaced by `what is probable'; it had
reached such a ridiculous point, he claimed, that innocent defendants in
court would neglect to say what actually happened, because it was better
to say something which would have seemed more likely to the
jurors.\footnote{\textsuperscript{} Plato, \emph{Phaedrus} (261)} Plato
and Isocrates both agreed, too, that sophists' claims to teach virtue
with their rhetorical education was misleading. For instance, the
argument that sophists would not have to complain, as they frequently
did, of injuries from their pupils -- unpaid wages, unfulfilled
contracts, and the like -- if they had actually taught their pupils to
be virtuous is one common to both Plato and Isocrates.\footnote{\textsuperscript{}
	Plato, \emph{Gorgias} (519); Isocrates, \emph{Against the Sophists}
	(6) } Some scholars have found Plato's criticisms on this subject to
be more compelling than Isocrates'. Plato was `more vehement', and
raised these criticisms more frequently than Isocrates who only took
issue with such methods when he was trying to distinguish himself from
other teachers and attract students.\footnote{\textsuperscript{} Benoit,
	1991} I note, however, that these criticisms recur in Isocrates'
\emph{Antidosis} published forty years later after \emph{Against the
	Sophist}. Both Plato and Isocrates then, set out to teach differently to
the sophists, and fiercely criticise other methods of rhetorical
education.

Nevertheless, Isocrates and Plato disagreed with each other on how the
truth and virtue lacking in their competitors' methods could be taught
to orators. While Isocrates does express concern for the truth, Plato is
much more ideologically strict. For Plato, truth is absolute and
irrefutable.\footnote{\textsuperscript{} Plato, \emph{Gorgias} (473)} We
must strive to reach that truth through dialectic, and any other method
is insufficient. So before any of the Academy's pupils could hope to
give a true speech (and speeches, as we have established, ought to be
true), he must have undergone a rigorous dialectic education. Isocrates
on the other hand did not think it was possible, or at least
realistically practical, to achieve certain knowledge on all
events.\footnote{\textsuperscript{} Isocrates, \emph{Against the
		Sophists} (2) } Instead, the student of rhetoric ought to use his
experience and conjecture to determine the likely truth of what he was
planning to say. Indeed, Isocrates proposes that such a method is
usually more consistent in reaching the truth than those who profess to
have exact knowledge.\footnote{\textsuperscript{} Isocrates, Antidosis
	(271)} So Isocrates wanted orators to be taught to make reasonable
judgements, but in Plato's view they ought to have a philosophically
rigorous dialectic education and reach the truth before applying
rhetoric to a topic.

This difference in thinking extended, too, to how the two thought
students of rhetoric could be virtuous. As former pupils of Socrates,
both Plato and Isocrates saw being virtuous as comprising at least in
part the cultivation of the \emph{psuche }(i.e. the metaphysical
component of our existence, usually translated as \emph{soul} or
\emph{mind}).\footnote{\textsuperscript{} Grote 1867} Both Plato and
Isocrates separated the \emph{psuche} from the physical body, and both
frequently analogised the two: both saw educating the mind as comparable
to, and even more important than, physical training and medicine for the
body. But to Plato an education in rhetoric in itself was of little
benefit to the \emph{psuche}. As cooking is to medicine, rhetoric is to
cultivating the \emph{psuche}: a `flattery' which provides some pleasure
but does not address what is actually good for one's health.\footnote{\textsuperscript{}
	Plato, \emph{Gorgias} (500)} But Isocrates was certain that his
pupils' \emph{psuche }benefited immensely from a rhetorical education,
because the aim of rhetoric is to persuade, and people are most
persuaded by those who are most virtuous in life.\footnote{\textsuperscript{}
	Isocrates, Anitdosis (276-8)} His pupils would, therefore, strive to
be virtuous in their lives in order to be more persuasive. Presumably
his school offered some instruction in which behaviours were moral (and
therefore persuasive). Indeed, he was so certain that his method of
education inspired virtue that he urged a hypothetical jury to convict
him at once if just one of his pupils was a bad man.\footnote{\textsuperscript{}
	Isocrates, \emph{Antidosis} (99-100)} Plato, too, thought rhetorical
teachers were to blame if their former pupils turned out as bad eggs:
Gorgias, one of Socrates' interlocutors in the eponymous dialogue, is
made to say unconvincingly that he is not to blame for his pupils'
misdemeanours.\footnote{\textsuperscript{} Plato, \emph{Gorgias} (456)}
To Plato however, what they needed was not the better rhetorical
education Isocrates thought he provided, but a foundation in Plato's
dialectic before they were taught rhetoric.

Plato saw the cultivation of the \emph{psuche} as the central goal of
all education, including in rhetoric. As the body must be kept healthy,
so must the \emph{psuche}: the consequences otherwise would be felt in
the afterlife when one is judged on the basis of the \emph{psuche} by
Minos and Rhadamanthus (those with the best \emph{psuche} would be sent
to the Isles of the Blessed; those with the worst to suffer in
Tartarus).\footnote{\textsuperscript{} Ibid } Since, as we have seen,
Plato did not think an education in rhetoric nourished the \emph{psuche}
in itself, rhetorical education should be pursued less than the
education in dialectic philosophy that would nourish the soul. Rhetoric
could be useful, however, to teach \emph{psuche}-nourishing things to
the layman, which may have been one of the reasons Plato allowed
lectures on rhetoric in his Academy (entire afternoons were even devoted
to the lectures given by Aristotle which became his treatise \emph{On
	Rhetoric}).\footnote{\textsuperscript{} \emph{Benoit}, \emph{Isocrates
		and Aristotle on Rhetoric}, 1990} Isocrates was not blind to the needs
of the \emph{psuche}, though, and as we have seen thought a proper
rhetorical education could help those `in the hands of the
Gods'.\footnote{\textsuperscript{}Isocrates, \emph{Antidosis} (281-2)}

Nourishment of the \emph{psuche} was not, however, Isocrates' only
reason for teaching rhetoric. He placed far more emphasis on the
practical political benefits of a polis full of great orators.\footnote{\textsuperscript{}
	Ibid (231-4; 316-8)} Throughout his works political achievements,
especially of Athens, are attributed to those well trained in rhetoric:
from Solon to Pericles, all the major political highpoints are ascribed
to `excellent orators'; conversely defeat in the Peloponnesian War, and
the Spartan occupation of the acropolis, along with other `great
misfortunes' are ascribed to speakers who are incorrectly educated and
so `full of insolence'. Isocrates dealt with what he saw as the most
pressing political concern, the urgency of Panhellenic unity,
consistently and vigorously throughout his career, and clearly hoped
that his pupils would too use the oratory he taught them to the good of
the polis. `Our every act,' he urged, should be `to enable us to govern
wisely our own household and commonwealth', and went so far as to say
that `those who ignore our practical needs' in their studies do not
deserve the title `students of philosophy'.\footnote{\textsuperscript{}
	Isocrates Antidosis (285)} (He was adamant that his own teaching of
rhetoric was `philosophy'.) Isocrates was so convinced of the importance
of the education he offered to the polis he reportedly did not charge
Athenian citizens to attend his school, living instead off fees from
foreigners and wealthy donors.\footnote{\textsuperscript{} Grote, 1867}
Plato's Academy, too, may not have charged citizens, and he is known to
have received donations from the same wealthy foreigners as
Isocrates.\footnote{\textsuperscript{} Ibid} His mission, however, was
much less public spirited than Isocrates. Disgusted by how the polis had
treated his mentor Socrates, Plato turned away from public politics and
devoted his teaching only to philosophy to enhance the individual
\emph{psuche}.\footnote{\textsuperscript{} Benoit, 1991}

The differences between Plato's and Isocrates' views on a rhetorical
education are perhaps best encapsulated in their differing approaches to
rhetorical style. Isocrates thought that style was essential to a good
speech. We have already seen his criticism of other rhetorical educators
for their failure to impart good style, instead teaching their pupils
stock phrases and unimaginative analogies. Isocrates consistently
treated good speeches like poetry.\footnote{\textsuperscript{}
	Isocrates, \emph{Antidosis} (47)} Indeed, a sense of poetry pervades
Isocrates' own works. Balanced clauses and antitheses give a rhythm that
some have criticised him for prioritising over clarity of meaning.
Plato, on the other hand, thought that stylistic form must follow
function. By cutting out redundant stylistic features the philosophical
meat was exposed, for instance when Socrates asks Gorgias to prioritise
brevity over his usual long style for the sake of productive
discourse.\footnote{\textsuperscript{} Plato, \emph{Gorgias} (449)} An
abundance of style, Plato felt, could obscure or distract from the
central issues of a speech -- even Socrates himself is distracted by
rhetorical acrobatics in the speech Phaedrus recites, and loses focus of
the main argument.\footnote{\textsuperscript{} Plato, Phaedrus (235)}
Plato did, however, put a speech in more exuberant style the mouth of
his hero Socrates in the dialogue \emph{Phaedrus}. Socrates mocks his
own overblown style, worrying he will soon be speaking in poetry and
making tongue in cheek parodies of epic.\footnote{\textsuperscript{}
	Ibid (237)} All this silliness, he complains, Phaedrus had foisted
upon him.\footnote{\textsuperscript{} Ibid (238)} This is because style
must, Plato felt, adapt to circumstance: Socrates could not persuade
Phaedrus with dry philosophical pronouncements. Isocrates agreed that
good speeches needed to adapt to circumstances and audiences to persuade
and entertain the audience.\footnote{\textsuperscript{} Isocrates,
	\emph{Against the Sophists} (16)} While Plato felt that students of
rhetoric had to be able to speak poetically when the occasion demanded
it, they learned to do so not because of the inherent value that
Isocrates saw in beautiful speeches, but because those speeches could be
in service of Plato's philosophy.

If Plato and Isocrates did not think pupils should study rhetorical
techniques for the same reasons, their views on who would be good at
studying them were notwithstanding remarkably aligned. They both agreed,
as discussed above, that knowledge is crucial for a good speech. In
order to become a truly `finished' orator, however, Plato suggested a
pupil needs both innate ability and practice.\footnote{\textsuperscript{}
	Plato, Phaedrus (269)} Isocrates agreed that `the practical experience
and innate ability of the student' were very important in becoming a
good performer.\footnote{\textsuperscript{} Isocrates, \emph{Against the
		Sophists} (10); \emph{Antidosis} (189)} He added, too, confidence,
(though that may be said to be part of innate ability) and even refused
to perform himself for fear he lacked presence, instead opting to
publish his would be speeches in (some of the world's first) political
pamphlets. Both Plato and Isocrates operated elite institutions; the
Academy was not open to all, and Isocrates limited the number of his
pupils to nine at a time.\footnote{\textsuperscript{} Plutarch, Lives of
	Ten Orators} Only those with the abilities they sought were educated
by both, despite -- or perhaps because of -- their respective missions
to enhance the \emph{psuche} and the polis.

Although at first glance they apparently reached very different
conclusions, Plato's and Isocrates' thoughts covered similar areas in
their considerations of a rhetorical education, and in much the same
way, and with the same result of founding a school in which their
methods could be taught. They were aligned in their criticisms of the
sophists. They were both concerned by the lack of truth in rhetoric, and
both discussed whether there might be epistemological benefits to an
education in speaking. Both, perhaps as a result of the influence their
shared teacher Socrates, considered the impact of rhetorical training on
the pupil's \emph{psuche}. Both were worried with the potentially
dangerous misuse of rhetoric, and the responsibilities of the teacher
for it. However, Plato did not see a training in rhetoric alone as
particularly valuable; his students focused on matters that would
enhance their \emph{psuche}. Isocrates saw an education centred on
rhetoric -- though encompassing a broader curriculum -- as one of the
key ways to nourish students' \emph{psuche}, as well as create valuable
works of poetic beauty, and -- most importantly -- address matters of
key concern to the polis. To Plato, an education in rhetoric was
secondary to his dialectic philosophy and at best a tool to be applied
(according to circumstance) to spread philosophical teachings. He even
has Socrates express the hope to Phaedrus that the promising young
Isocrates abandons his current course and turns towards proper dialectic
philosophy.\footnote{\textsuperscript{} Plato, Pheadrus (279)} (A hope
that the audience knows is unfulfilled.) To Isocrates, however, an
education in speaking \emph{was} philosophy, and philosophy of a more
valuable kind than one without practical application. Most importantly
however, Plato and Isocrates agreed on the bigger picture of a
rhetorical education. Both Plato and Isocrates presided over the first
prominent institutions teaching rhetoric to a select group of pupils
selected based on their talent and proficiency for hard work. They both
recognised that if would-be pupils of rhetoric were to be taken out of
the hands of the sophists and receive an education on what they thought
mattered (the benefit of the \emph{psuche} and polis respectively), they
needed to found institutions that taught rhetoric.

\end{multicols}

\begin{refsection}

\clearpage

\nocite{CiceroOratore1942}
\nocite{contrasophists}
\nocite{IsocratesDiscourses15Antidosis1929}
\nocite{PlatoGorgias1925}
\nocite{BenoitIsocratesAristotleRhetoric1990}
\nocite{BenoitIsocratesPlatoRhetoric1991}
\nocite{PlatoPhaedrus1914}
\nocite{GrotePlatoothercompanions1867}
\nocite{DionysiusofHalicarnassusAncientOratorsIsocrates1974}
\nocite{QuintilianOratorEducation2002}

\section{Bibliography}

\begin{multicols}{2}

\textbf{It should be noted that the visits noted herein were the Editor's, in the typesetting process. They are preserved because part of the utility of the inclusion of such dates is that they inform readers of the likelihood of their availability when they later read citations; it is unlikely that any of these texts should change at their respective web addresses, for they are copies of texts written, in the case of the originals, millennia ago.}

\printbibliography[heading=none]

\end{multicols}

\end{refsection}