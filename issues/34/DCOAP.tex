The broad thesis of the first of this series of articles is that China
is unlikely to liberalise in the near future, and that the most likely
path is one of gradually increasing oppression. This article will
explain why China is increasingly able to implement such goals, both at
home and abroad.

A broad effect of modernity is that everything has become bigger in
magnitude: people, numbers of people, armies, weapons, speeds, data and
coercive capacity have all massively increased in scale, complexity and
efficacy. Hegemony in the tenth century was local, and necessarily so.
Hegemony with modernity affects nearly everyone.

The events predicted in this article are likely to affect vast numbers
of people outside China. They will not necessarily happen instantly, and
there may be reversals within the broad trends described. Nevertheless
the author has a relatively high degree of confidence in the general
direction described.

\subsection{Domestic control}\label{domestic-control}

\subsubsection{Propaganda}\label{propaganda}

Control of information may not change individual axiology but it
certainly changes individuals' decisions as to how best to implement
that axiology. We should not deny the former, for many of our
instrumental beliefs, and indeed core axiological beliefs, are imparted
by the outside world, and so are influenced by those who control access
to information. Nevertheless, the primary concern of the modern day
Chinese censor is not to change axiological beliefs in the value of
family, riches or success. Rather, it is to change the instrumental
values which are perceived to be the best way to achieve them. Xinhua's
press release on Xi Jinping thought claims that such thought is an
attempt to become "a great modern socialist country that is prosperous,
strong, democratic, culturally advanced, harmonious, and beautiful by
the middle of the century."\footnote{http://news.xinhuanet.com/english/2017-10/24/c\_136702802.htm}

Propaganda's value is derived from its ability to change thought.
Previously, the Chinese censor was constrained by a need to retain some
credibility. The efficacy of Chinese propaganda was necessarily limited
by the tradeoff between message and credibility.

China, however, has moved on from real life. The rise of a peculiarly
Chinese internet has been the primary cause of what has happened. It is
unique in two ways: first, the quality and quantity of its adoption, and
second, the degree of state control over it.

\subsubsection{Information control}\label{information-control}

China has the world's most internet users - 732 million\footnote{http://www.scmp.com/tech/china-tech/article/2064396/chinas-internet-users-grew-2016-size-ukraines-population-731-million}.
That China's total population is 1.4 billion masks something important -
almost all young people use the internet. The quality of China's
internet usage reflects two important characteristics of this growth.
First, its magnitude incentivises internet development in a way which
smaller markets elsewhere do not. Second, that most of this growth was
recent meant that the way that the internet is used in China is
primarily mobile-led and significantly more adapted to the latest
technologies - epayments and so on - than other markets.

Chinese internet usage is therefore different to that in the West. It
includes much more than its Western equivalents - payment systems for
not only online shopping but also physical shopping, a greater embrace
of the ``gig'' economy in providing platforms for freelancers to sell
their services, and heavy integration between and within apps are
particularly salient features of one representative and popular app -
WeChat\footnote{https://www.economist.com/news/business/21703428-chinas-wechat-shows-way-social-medias-future-wechats-world}.
\emph{The Economist} claims that the unique characteristics of the
Chinese market - its size, a propensity to possess multiple devices and
the cultural phenomenon of red packets - were part of this shift.
Another important factor is, however, that the Chinese internet is
deeply separated from the rest of the world. Although one can reach the
outside world from the inside, and outsiders can reach the Chinese
internet, such connexions are often slow - a number of senior Chinese
political figures have complained about this, as have
scientists\footnote{https://www.hongkongfp.com/2017/03/19/chinese-scientists-speak-great-firewall/}.
The connexion between China and the rest of the world is akin to having
a dirt path between two major cities - though there may be few
insurmountable obstacles, most would not attempt to use it. Users are a
little like drivers, in that they will quickly abandon that which is
inconvenient\footnote{https://www.thinkwithgoogle.com/data-gallery/detail/mobile-site-abandonment-three-second-load/}.
That internet users are fickle means that they are very likely to use
Chinese tools instead - especially since Western tools now do not appear
to offer any particular benefit over their Chinese equivalents.

Chinese control over the internet broadly sues the same technologies
which are available to Western countries - filtering, vast numbers of
administrators and moderators, and other heuristics-based algorithms
which block terms. Two things differ: first, political will, and second,
the overtness with which the Chinese government controls the internet.
In the West, perhaps the most draconian régime is that of the United
Kingdom, where § 58 of the Terrorism Act (2000) states:

\begin{enumerate}
\item
  A person commits an offence if---

  \begin{enumerate}
  \item
    he collects or makes a record of information of a kind likely to be
    useful to a person committing or preparing an act of terrorism, or
  \item
    he possesses a document or record containing information of that
    kind.
  \end{enumerate}
\item
  In this section ``record'' includes a photographic or electronic
  record.
\item
  It is a defence for a person charged with an offence under this
  section to prove that he had a reasonable excuse for his action or
  possession.
\item
  A person guilty of an offence under this section shall be liable---

  \begin{enumerate}
  \item
    on conviction on indictment, to imprisonment for a term not
    exceeding 10 years, to a fine or to both, or
  \item
    on summary conviction, to imprisonment for a term not exceeding six
    months, to a fine not exceeding the statutory maximum or to both.
  \end{enumerate}
\end{enumerate}

We should note, of course, that a number of potentially ``reasonable''
excuses have been rejected by the courts, including attempting to
understand terrorist ideology\footnote{https://www.theguardian.com/world/2012/dec/06/woman-jailed-al-qaida-material-on-phone}.

In China, control goes much further. Not only is the scale of censorship
greater, but that there is censorship is censored. Not only are
``terrorist'' materials proscribed - those advocating democracy or
attacking the régime in any way whatsoever are prohibited\footnote{https://www.economist.com/news/special-report/21574631-chinese-screening-online-material-abroad-becoming-ever-more-sophisticated}.

Willingness to politicise the internet - to use its technical
capabilities in a political fashion as opposed to using its technical
capabilities apolitically to facilitate political actions, combined with
its pervasive influence over Chinese lives, is

Any claim that China will liberalise because its people demand it must
overcome this new reality of information control in China. The Communist
Party are now here to stay - they have always had a desire to, and now
they have the power to.

\subsection{International control}\label{international-control}

China is increasingly able to assert its power abroad. This occurs in
two ways. First, China increasingly has a strategic advantage which
enables it to assert its power over other nations. Second, Chinese
people, in occupying positions elsewhere, are often vulnerable to
mainland influence in a number of ways - cultural, political, familial
and economic.

\subsubsection{China's strategic
advantage}\label{chinas-strategic-advantage}

China has, broadly, five advantages, primarily over the United States,
but also over the rest of the world.

First, China has the world's largest manufacturing base\footnote{https://data.worldbank.org/indicator/NV.IND.MANF.CD?locations=CN-US-JP}.
Its advantage here is especially pronounced in the field of electronics.
Informatic systems are important in all fields of life - most Western
economies rely on them to function. No other country is able to match
China in quantity, and, increasingly, quality. China will be able to
exploit this to its advantage as it increasingly becomes less dependent
on other nations for development, as exports as a proportion of its GDP
have decreased\footnote{https://data.worldbank.org/indicator/NE.EXP.GNFS.ZS?locations=CN}
and are likely to decrease further.

Second, China controls the natural resources critical for such
manufacture. China supplies 85\% of the world's rare earths\footnote{https://thediplomat.com/2017/08/revisiting-rare-earths-the-ongoing-efforts-to-challenge-chinas-monopoly/}.
These are crucial in building informatic systems.

Third, China has a population advantage, especially when compared to
other potential competitors. Russia and the United States have
significantly smaller populations at a global level. At a local level,
only India can match its population; all other nations which China will
seek to influence in the coming years - Japan, South Korea, and those in
South East Asia, have far smaller populations.

Fourth, China has a capacity for coördinated action which few other
nations possess. It is relatively trivial to see that American
dysfunction is unlikely to disappear in the near future. India, too,
despite BJP political hegemony, retains its everlasting capacity to slow
reforms down. This gives it an advantage over other nations in adopting
new technology, upgrading its military, increasing economic growth with
infrastructure projects, maintaining domestic political control and
increasing domestic production for strategic purposes.

Fifth, China increasingly leads the world technologically. Consortia of
Chinese scientists have been particularly important in recent
developments in quantum physics\footnote{http://www.bbc.co.uk/news/science-environment-40294795}.
They have also been at the forefront of genetic research\footnote{http://www.nature.com/news/chinese-scientists-to-pioneer-first-human-crispr-trial-1.20302}.

\subsubsection{People}\label{people}

There are 50 million Chinese abroad\footnote{http://www.asiapacific.ca/sites/default/files/filefield/researchreportv7.pdf}.
Many of these Chinese occupy important positions in academia, politics,
law and so on. They are particularly prominent in Singapore and South
East Asia as a whole, where their traditional prosperity has often made
them targets for political oppression, as occurred in Indonesia under
Suharto.

Most Chinese have relatives in China. China has no compulsion against
attacking relatives\footnote{http://www.telegraph.co.uk/news/2016/03/28/another-chinese-dissident-says-relatives-detained-over-letter-cr/}\textsuperscript{,}
\footnote{http://www.reuters.com/investigates/special-report/china-uighur/}.
Given this, it appears likely that China will start to, if it has not
already done so, use its overseas Chinese in dubious ways.

This presents an important moral problem. Many Chinese live abroad. It
would normally be abhorrent to impose a blanket ban on the employment of
Chinese in sensitive positions. Nevertheless, when the assumption that
would justify such a ban - that Chinese people cannot be trusted - is
undermined (in this case, by the Chinese government), it is unclear
which should be prioritised. We should expect to increasingly be faced
with such problems.

Chinese people in public life who are not spies will also face problems.
There will be a permanent cloud of suspicion over their heads.

\subsubsection{A collective action
problem}\label{a-collective-action-problem}

It is clear that China has a vast number of carrots to offer other
nations in attempting to impose its will. There is, of course, a classic
collective action problem here. Those who refuse to coöperate on a
principled basis lose, but the Chinese people gain nothing from such
refusals, for China will always find someone else to interact with.

Coördinated opposition could come from an alliance of relatively large

\subsubsection{Winds of change}\label{winds-of-change}

We underestimate the power that a sense that ``progress'' is one way or
another has. During the end of history, we saw significant
liberalisation and democratisation. This correlated significantly with
two phenomena: first, the fall of the Soviet Union and hangers-on, and
second, the victory of the idea of liberal democracy.

As the Soviet Union fell, dictatorships around the world loosened
controls or fell. In Russia, there was, for a period, genuine freedom of
expression, and, to some degree, liberty. There is evidently none now.
Empirically, there is a correlation.

Why? All dictatorships need a raison d'être, even if merely for show.
This is why dictatorships which have complete control or near complete
control over their societies - the DPRK, China and Russia, for example -
still require some explanation for their actions. No dictatorship has
yet openly declared itself a kleptocracy. This is for two reasons.

First, officials, when overly kleptocratic, cause state collapse, by
overly extracting rent. This can be solved with an ideological basis -
it ensures lip service, allows enforcement mechanisms to be
significantly more effective, and motivates officials in a way which
overt kleptocracy does not.

Second, there is an innate human desire to justify one's actions. It is
possible that humans are incompatible with overt kleptocracy. Even some
of the most evil people that the world has ever seen - Hitler, Stalin
and Mao, attempted to frame their actions as conducive to a certain
(often warped) perception of ``the good''.

It is most certainly not for want of material control that this occurs.

China's dominance means that the winds of progress are perceived to blow
towards it. In the West, confidence in democracy has declined\footnote{https://www.weforum.org/agenda/2016/12/charts-that-show-young-people-losing-faith-in-democracy/}.
Outside it, dictators who previously claimed that the path to democracy
was simply to be had later\footnote{https://www.thetimes.co.uk/article/iraqis-not-ready-for-democracy-says-blairs-envoy-qkf8rm6j3wf}.
Now, it has grown - dictators no longer need to pay lip service to it,
and they now feel free to entirely reject, down to an axiological level,
democracy.

This is important because even though lip service does not always result
in significant gains in freedom for the majority of the population, the
groups which it allows to exist are important in any post-dictatorship
scenario. That, for example, there was continuity between colonial and
postcolonial institutions in many nations appears to have contributed to
stability in a number of British colonies. Where there was no continuity
or this continuity was rejected, instability reigned. We may, for
example, compare the Democratic Republic of the Congo and Namibia. This
is not exactly a fair comparison for a number of reasons, but
empirically nevertheless there is something to be said for continuity.

When dictators feel free to destroy civil society, there is nothing
left. Societies may genuinely be better off with dictators than under
anarchy. Those who would resist do not.

Resistance to oppression involves another collective action problem. If
every single North Korean not part of the régime were to attempt to
overthrow it, the régime would soon fail. Yet this does not happen. For
every individual, the price of resistance is very great, and the likely
gain very limited. Once the collective action problem is overcome,
whether by chance or particularly egregious oppression, there is no more
problem. Civil society organisations are crucial in overcoming
collective action problems. They allow resistance to régimes to
crystallise, providing a focal point for others to join and a core
organisation.

China's ascendency empowers dictators to copy its model, even if it does
not take an active interest.

\subsection{Conclusion}\label{conclusion}

China's present leadership has the tools to stay in power for a very
long time, both internally, with its control of information, and
externally, with its strategic advantages. As demonstrated in the first
part of this article, it is not a particularly pleasant one. Its model
of dictatorship will be exported, not primarily deliberately but simply
by its changing of the international political climate. As it does,
other nations will also be able to acquire the tools to entrench their
leaders just as China has done.
